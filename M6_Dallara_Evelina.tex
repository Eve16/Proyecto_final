\documentclass[a4apaper,twocolumn,10pt]{article}
\usepackage[spanish]{babel}
\usepackage[utf8]{inputenc}
\usepackage{graphicx}
\usepackage{flushend}
\usepackage{textcomp}
\usepackage{amssymb, amsmath, amsbsy} 
\usepackage{hyperref}

\begin{document}
\title{Interacción de las erupciones
volcánicas con la atmósfera:
interacciones específicas y casos históricos}
\author{Evelina Dallara}
\date{}
\twocolumn[
\begin{@twocolumnfalse}
\maketitle
\begin{abstract}
Repositorio: \href{url}{https://github.com/Eve16/Proyecto\_final.git}
\\Palabras clave: Erupciones volc\'anicas, atm\'osfera, capa de aerosoles de Junge
\end{abstract}
\end{@twocolumnfalse}
]

\section{Introducción}
Las erupciones volcánicas tienen un efecto importante tanto a escala local como global,
considerando la superficie terrestre como la atmósfera. Los efectos sobre esta última pueden
causar cambios importantes en el clima, a pequeños y grandes intervalos de tiempo, afectando
de manera diferente a la sociedad. Las interacciones entre la columna eruptiva y la atmósfera
se pueden observar de distintas formas, en este caso vamos a considerar principalmente la capa
de aerosoles de Junge y la formación de las nubes cirrus. Además, se van a presentar distintos
casos de erupciones históricas que han tenido impactos importantes a escala global. Estos son
la erupción del volcán Tambora de 1815 y la del Pinatubo en 1991, y luego se presenta un
enfoque distinto al considerar una región (Polonia) con un recorrido temporal evidenciando los
eventos volcánicos más importantes.
\section{Efectos sobre la capa de aerosoles de Junge}
La capa de aerosol de Junge es un estrato global situado a unos 20 km de altitud que refleja la luz solar y por tanto lleva a un enfriamiento de la atm\'osfera inferior, as\'i como a un calentamiento local por la absorci\'on de la radiaci\'on \cite{von2009effects}. Se ha podido observar como las erupciones explosivas que llegan a alcanzar esta altitud tienen un importante impacto sobre esta \'ultima capa, debido mayormente a la emisi\'on de sulfuros. Estos sulfuros, emitidos en forma de $SO_{2}$ se oxidan formando el $H_{2}SO_{4}$, que puede llevar a un aumento de los aerosoles de sulfato l\'iquido y que pueden permanecer en la atmósfera durante años
\cite{vernier2011major}. La oxidaci\'on de este \'ultimo se produce seg\'un las ecuaciones 
\begin{displaymath}
OH+SO_{2}\rightarrow HSO_{3}
\end{displaymath}
\begin{displaymath}
HSO_{3}+O_{2}\rightarrow SO_{3}+HO_{2}
\end{displaymath}
\begin{displaymath}
SO_{2}+2H_{2}O\rightarrow H_{2}SO_{4}+H_{2}O
\end{displaymath}
Uno de los ejemplos más significativos del efecto de una erupción volcánica sobre la
capa de Junge ha sido la erupción del Pinatubo en 1991. Durante esta erupci\'on los 20 Tg de $SO_{2}$ que habían sido emitidos llegaron a la capa de Junge y causaron un enfriamiento de la
temperatura global de unos 0.4 \textdegree C en el a\~no después de la erupción y un calentamiento de la estratosfera inferior de 1.5 \textdegree C. Además, después de esta erupción, los aerosoles volcánicos han sido transportados hacia latitudes mayores donde se produjeron reacciones que llevan a la formación del agujero del ozono. Con el tiempo los aerosoles disminuyen debido a
sedimentación e intercambios entre la estratosfera y troposfera, así que la estratosfera llega de
nuevo a estar en condición no-volcánica. De hecho, en ausencia de erupciones volcánicas la
presencia de la capa de Junge se relaciona con la emisión en la superficie de precursores de
gases sulfúricos
\newpage

\bibliographystyle{apalike}
\bibliography{Bibliografia_6.bib}
\end{document}